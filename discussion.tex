\documentclass[11pt]{article}
\usepackage{fullpage,amsmath,amsfonts,mathpazo,microtype,nicefrac,graphicx,verbatimbox,listings,hyperref,enumitem,amssymb,float,pdfpages}
% Macro definitions
\newcommand{\N}{\mathbb{N}}
\newcommand{\Z}{\mathbb{Z}}
\newcommand{\Q}{\mathbb{Q}}
\newcommand{\R}{\mathbb{R}}
\newcommand{\B}{\mathbb{B}}
\newcommand{\p}{\partial}
\newcommand{\Trans}{\mathsf{T}}
\renewcommand{\vec}[1]{\mathbf{#1}}
\newcommand{\vx}{\vec{x}}
\newcommand{\vb}{\vec{b}}

\DeclareMathOperator{\rank}{rank}

% Include support for including a pdf

% \begin{lstlisting}[frame=single]
% \end{lstlisting}


% \begin{figure}[H]
% \centering
% \includegraphics[keepaspectratio=true,scale=0.6]{problem3d}
% \caption{Plot of temperature cross section in pipe at various time-steps of simulation}\label{visina8}
% \end{figure}

\begin{document}

\lstset{language=Python, basicstyle=\scriptsize} % set language as python

\section*{AM205 Project - Traffic Flow Modelling; do autonomous cars help? (I don't know about this but we can change it)}

\section{Introduction}
\paragraph{}The modeling of traffic at traffic lights, on highways, in dense streams etc. can often present a problem that is too complicated to solve analytically. This is mainly due to the inherent random movements of cars, the number of individually moving cars and the many scenarios that can be analyzed. These scenarios can be modeled numerically by applying the equations of motion to a system of vehicles and observing the impact of a number of outcomes. In this project, we will present an analytical solution to a number of simplistic scenarios involving traffic (cars starting from a red light, cars approaching a red light etc). We will then present the same scenario using a numerically calculated solution, modeling the cars using a differential setting where one car's motion is dependent on the car that is in front of it. We will then extend the scope of this analysis and introduce a number of varying factors into the model to analyze the motion of cars along a straight road, analyze the cars along a circular track and finally include autonomous cars with the human controlled vehicles. These autonomous cars have more precise, and faster reacting velocity controllers and the inclusion of vehicles such as these in the model is expected to smooth the flow of traffic and ultimately increase the density of traffic flow.

\section{Macroscopic Perspective of Traffic Interaction}
\paragraph{}There are generally two ways to model traffic. In the macroscopic perspective, traffic is viewed as a fluid or gas with a given maximum density moving according to the laws of conservation of mass. Because we'd like to interweave different models for specific cars, these models would be difficult to blend in a way that is easily interpretable. 

\section{Microscopic Perspective of Traffic Interaction : \\Intelligent Driver Models}
\paragraph{}In the microscopic perspective, individual cars are modeled as particles that move according to a relationship to their leading particle(s).
For this section of the problem, we will model the behaviors of cars in a stream of traffic individually. We will then simulate the movement of this stream of cars and demonstrate some well known traffic phenomenon (cars starting from a red traffic light will propagate forward in a wave formation, cars traveling along a straight road will oscillate into a traffic jam). Furthermore, we will use different controllers to simulate the means by which cars can control their acceleration and will be able to simulate the effect that autonomous cars may have for the throughput of a stream of traffic for varying car densities. 

$$\dot{v} = a \left[1 - \left(\frac{v}{v_0}\right)^{\delta} - \left(\frac{s*(v,\Delta v)}{s}\right)^{2}\right]$$

$$s*(v, \Delta v) = s_0 + max\left(0, vT + \frac{v \Delta v}{2 \sqrt{ab}} \right)$$

$$a_{free}(v)= \begin{cases}
a \left[ 1 - (\frac{v}{v_0})^\delta \right] & v \le v_0\\
-b \left[ 1 - (\frac{v_0}{v})^{a\delta/b} \right]& v > v_0
\end{cases}$$

$$\frac{dv}{dt}\Bigr|_{v\le v_0}= \begin{cases}
a (1-z^2) & z = \frac{s*(v, \Delta v}{s} \ge 1\\
a_{free}(1 - z^{(2a)/a_{free}})& otherwise
\end{cases}$$

$$\frac{dv}{dt}\Bigr|_{v> v_0}= \begin{cases}
a_{free} + a (1-z^2) & z(v, \Delta v) \ge 1\\
a_{free} & otherwise
\end{cases}$$

$$a_{CAH}(s,v,v_l, \dot{v}_l)= \begin{cases}
\frac{v^2\tilde{a}_l}{v_l^2 - 2 \tilde(a)_l} & v_l(v-v_l) \le -2s\tilde{a}_l\\
\tilde{a}_l - \frac{(v-v_l)^2 \Theta (v-v_l)}{2s} & otherwise
\end{cases}$$

$$\tilde{a}_l(\dot(v)_l) = min(\dot{v}_l, a)$$

$$a_{ACC}= \begin{cases}
a_{IIDM} & a_{IIDM} \ge a_{CAH}\\
(1-c)a_{IIDM} + c\left[a_{CAH} + b tanh(\frac{a_{IIDM}-a_{CAH}}{b}) \right] & otherwise
\end{cases}$$

%This numerical analysis can be contrasted to the analytical solution that was presented in (**another section 2?**).\par
%A car`s speed is governed by the laws of motion:
%\begin{equation}
%v_f = v_i + a\Delta t
%\end{equation}
%
%An individual car can be modeled as such over a 100,000m track and asked to travel at $v_{target} = 60miles.h^{-1}$ (i.e. $26.8m.s^{-1}$). As the driver is not an perfect controller he is unable to hold the speed exactly at $v_{target}$. We can therefore model his fluctuations as random acceleration with the corresponding adjustments to correct the acceleration:
%\begin{equation}
%a_{correction_i} = \alpha(v_{instant} - v_{target}) + N(0,\sigma_1)
%\end{equation}
%
%We note that for more than one car in a stream of traffic the car's acceleration is then also dependent on the car ahead of it:
%\begin{equation}\label{eq:interaction}
%a_{interaction_i} = \beta (s_{instant} - s_{target})
%\end{equation}
%
%In equation \ref{eq:interaction}, we see that the acceleration of the car is also dependent on the distance between any one car and the car that is directly ahead of it. The final acceleration of the car is therefore:
%\begin{equation}\label{eq:acceleration}
%a_{i} = a_{correction_i} + a_{interaction_i}
%\end{equation}
%
%If we assume the cars should travel at $v_{target}$ and that they should maintain an $s_{target} = 5m$ spacing between them, we can then model the velocity of the car as:
%
%\begin{equation}\label{eq:velocity}
%v_{i} = \frac{d(a_i)}{dt}
%\end{equation}

\section{Microscopic Perspective of Traffic Interaction : Human Driver Models}

\section{Numerical Simulations of Traffic Systems}

\paragraph{}To generate numerical simulations of Traffic Systems, we first consider the case of 50 cars on a circular track. We explored scenarios of all IDM cars and all HDM cars. Under these conditions, we see the IDM models generate more `phantom` traffic waves than the HDM models, especially for larger numbers of HDM car lookahead. This is because the human drivers tend to appear more conservative in a situation like this, with no external perturbations, as the multi-car lookahead prevents them from getting overly close to the car directly in front of them.

% \begin{figure}[H]
% \centering
% \includegraphics[keepaspectratio=true,scale=0.6]{problem3d}
% \caption{Plot of temperature cross section in pipe at various time-steps of simulation}\label{visina8}
% \end{figure}

% \begin{figure}[H]
% \centering
% \includegraphics[keepaspectratio=true,scale=0.6]{problem3d}
% \caption{Plot of temperature cross section in pipe at various time-steps of simulation}\label{visina8}
% \end{figure}

\paragraph{}We then moved on to simulations of a platoon of cars with some cars being controlled by IDM-type models (in particular, Adaptive Cruise Control) and some cars being controlled by multi-car lookahead HDM models. After allowing the models to run for some time to develop their natural spacing, we then perturbed the acceleration of the lead car, causing it to brake and then resume again. We then observe both the magnitude and length of the volatility of the velocities in the following cars after the event. We find that in general, as one would expect, a larger proportion of IDM-controlled vehicles results in a quicker dissipation of the perturbation as well as a lowered probability of a vehicle collision.
\textbf{RESULTS HERE}

\section{Stability and Error Comparison of a Variety of Numerical Integration Schemes}
Because there is no analytical solution given for our highly-nonlinear problem, to analyze the error of our numerical integration schemes, we compared to the python \texttt{odeint} solver function. For large enough time steps, this shows us error decreasing on a log scale for each of our higher order methods. However, for small time steps, the `error` for 4th and 5th order methods converges, as the \texttt{odeint} solver presumably decides that 4th order is sufficiently accurate at these time steps and stops using higher order methods.
\textbf{RESULTS HERE}
For our purposes, we desired to use higher order methods such that we could use larger timesteps and hopefully decrease running time of our programs while still maintaining stability in the differential equation systems. To display the effectiveness of these higher order methods, we ran one of our simple models for a variety of time steps over the same total time with the different solvers and noted the number of time steps at which each method stabilized.
\textbf{RESULTS HERE}

\section{String Stability of Traffic Systems to Perturbation}

%as a function of the reaction time Tr and the attention span ?t depends strongly on
%the number of considered vehicles. When considering na = 5 leaders, the critical
%effective reaction time Tr + ?t/2 at the stability limit is about twice as large as
%the corresponding critical value without multi-anticipation (na = 1). Particularly,
%traffic can be stable even if the reaction time exceeds the average time headway. This
%agrees with everyday observations but cannot be realized in simulations without
%multi-anticipation. There are limits, however: Anticipating more than five vehicles
%ahead will change the dynamics insignificantly

\section{Optimization of Traffic Systems}
\paragraph{}We set up a number of optimization programs to attempt to search for the optimal values of some of the many parameters for the system as a whole and for individual drivers for a number of cost functions, including maximizing displacement and maximizing comfort (minimizing jerk). The initial trials for these systems were run on IDM models, and the results are largely what one would suspect. When attempting to maximize total displacement, the cars drive quickly and close together, resulting in the creation of traffic waves in the system. When attempting to minimize jerk, the cars drive much more conservatively, as one would expect, never coming within the range of another car that would cause them to change acceleration. 
\textbf{RESULTS HERE}
\paragraph{}We then attempted to optimize behavior for a single driver and were interested to see what implications this would have for the system as a whole. However, unsurprisingly, the following behaviors in the IDM are such that a single driver can tailgate and maximize target speed basically as much as we will allow without ever disrupting the rest of the system. Even when adding some stochasticity to the lead car, we were unable to see the rest of the system suffer from traffic waves due to the behavior of the second car, because that behavior would simply be quickly dissipated by the third car. In fact, it seems to be true generally of these systems that unless the aggressive driving behavior of a car causes other cars to collide behind it, the greater distance achieved by one car inevitably results in other cars also achieving a greater distance/average velocity. However, this is likely to occur with much greater volatility in the velocity of the cars, i.e. a less pleasant ride. Then, to quantify how the erratic driving of a single car may result in penalties to the system as a whole but not to that car, we must analyze the cost to the other cars in terms of the volatility of their ride speed and possible collisions.\\
\textbf{RESULTS HERE}
\paragraph{}Next, we attempted to replicate these experiments with the more complicated Human Driver Model. However, due to the highly piecewise nature of the functions we are attempting to optimize over, we were unable to use traditional minimization solvers. Instead, we discretized the parameter space we were interested in exploring and searched for the set of parameters which would minimize our target cost function.
\textbf{RESULTS HERE}
\section{TODO}
develop a cost function for emissions/energy usage and optimize over that \\
can we find the parameters that will cause a phantom traffic jam?

\end{document}










